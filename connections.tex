\documentclass[aspectratio=169]{beamer}

\mode<presentation>
{
  \usetheme{Warsaw}
  % or ...

  \setbeamercovered{transparent}
  % or whatever (possibly just delete it)
}


\usepackage[english]{babel}
\usepackage[latin1]{inputenc}
\usepackage{graphicx}
%\usepackage{times}
%\usepackage[T1]{fontenc}
% Or whatever. Note that the encoding and the font should match. If T1
% does not look nice, try deleting the line with the fontenc.

\usepackage{amsmath,amsfonts,amssymb}

\newcommand{\pkg}{\textbf}
\newcommand{\code}{\texttt}


\title[The R Language]{Introduction to the R Language}

\subtitle{Connections}

\date{Computing for Data Analysis}

\setbeamertemplate{footline}[page number]


\begin{document}

\begin{frame}
  \titlepage
\end{frame}


\begin{frame}{Interfaces to the Outside World}
Data are read in using \textit{connection} interfaces.  Connections
can be made to files (most common) or to other more exotic things.
\begin{itemize}
\item
\code{file}, opens a connection to a file
\item
\code{gzfile}, opens a connection to a file compressed with gzip
\item
\code{bzfile}, opens a connection to a file compressed with bzip2
\item
\code{url}, opens a connection to a webpage
\end{itemize}
\end{frame}


\begin{frame}[fragile]{File Connections}
\begin{verbatim}
> str(file)
function (description = "", open = "", blocking = TRUE, 
          encoding = getOption("encoding"))
\end{verbatim}
\begin{itemize}
\item
\code{description} is the name of the file
\item
\code{open} is a code indicating
\begin{itemize}
\item
``r'' read only
\item
``w'' writing (and initializing a new file)
\item
``a'' appending
\item
``rb'', ``wb'', ``ab'' reading, writing, or appending in binary mode
(Windows)
\end{itemize}
\end{itemize}
\end{frame}


\begin{frame}[fragile]{Connections}
In general, connections are powerful tools that let you navigate files
or other external objects.  In practice, we often don't need to deal
with the connection interface directly.
\begin{verbatim}
con <- file("foo.txt", "r")
data <- read.csv(con)
close(con)
\end{verbatim}
is the same as
\begin{verbatim}
data <- read.csv("foo.txt")
\end{verbatim}
\end{frame}

\end{document}
