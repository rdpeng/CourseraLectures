\documentclass[aspectratio=169]{beamer}

\mode<presentation>
{
  \usetheme{Warsaw}
  % or ...

  \setbeamercovered{transparent}
  % or whatever (possibly just delete it)
}


\usepackage[english]{babel}
\usepackage[latin1]{inputenc}
\usepackage{graphicx}
%\usepackage{times}
%\usepackage[T1]{fontenc}
% Or whatever. Note that the encoding and the font should match. If T1
% does not look nice, try deleting the line with the fontenc.

\usepackage{amsmath,amsfonts,amssymb}

\newcommand{\pkg}{\textbf}
\newcommand{\code}{\texttt}


\title[The R Language]{Introduction to the R Language}

\subtitle{Vectorized Operations}

\date{Computing for Data Analysis}

\setbeamertemplate{footline}[page number]


\begin{document}

\begin{frame}
  \titlepage
\end{frame}


\begin{frame}[fragile]{Vectorized Operations}
Many operations in R are \textit{vectorized} making code more
efficient, concise, and easier to read.
\begin{verbatim}
> x <- 1:4; y <- 6:9
> x + y
[1]  7  9 11 13
> x > 2
[1] FALSE FALSE  TRUE  TRUE
> x >= 2
[1] FALSE  TRUE  TRUE  TRUE
> y == 8
[1] FALSE FALSE  TRUE FALSE
> x * y
[1]  6 14 24 36
> x / y
[1] 0.1666667 0.2857143 0.3750000 0.4444444
\end{verbatim}
\end{frame}

\begin{frame}[fragile]{Vectorized Matrix Operations}
\begin{verbatim}
> x <- matrix(1:4, 2, 2); y <- matrix(rep(10, 4), 2, 2)
> x * y       ## element-wise multiplication
     [,1] [,2]
[1,]   10   30
[2,]   20   40
> x / y
     [,1] [,2]
[1,]  0.1  0.3
[2,]  0.2  0.4
> x %*% y     ## true matrix multiplication
     [,1] [,2]
[1,]   40   40
[2,]   60   60
\end{verbatim}
\end{frame}


\end{document}
