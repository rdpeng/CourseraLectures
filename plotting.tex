\documentclass[aspectratio=169]{beamer}

\mode<presentation>
{
  \usetheme{Warsaw}
  % or ...

  \setbeamercovered{transparent}
  % or whatever (possibly just delete it)
}


\usepackage[english]{babel}
\usepackage[latin1]{inputenc}
\usepackage{graphicx}
%\usepackage{times}
%\usepackage[T1]{fontenc}
% Or whatever. Note that the encoding and the font should match. If T1
% does not look nice, try deleting the line with the fontenc.

\usepackage{amsmath,amsfonts,amssymb}

\newcommand{\pkg}{\textbf}
\newcommand{\code}{\texttt}


\title[The R Language]{Introduction to the R Language}

\subtitle{Plotting}

\date{Computing for Data Analysis}



\begin{document}

\begin{frame}
  \titlepage
\end{frame}


\begin{frame}{Plotting}
The plotting and graphics engine in R is encapsulated in a few base
and recommend packages:
\begin{itemize}
\item
\pkg{graphics}: contains plotting functions for the ``base'' graphing
systems, including \code{plot}, \code{hist}, \code{boxplot} and many
others.
\item
  \pkg{lattice}: contains code for producing Trellis graphics, which are
  independent of the ``base'' graphics system; includes functions like
  \code{xyplot}, \code{bwplot}, \code{levelplot}
\item
  \pkg{grid}: implements a different graphing system independent of the
  ``base'' system; the \pkg{lattice} package builds on top of
  \pkg{grid}; we seldom call functions from the \pkg{grid} package
directly
\item
\pkg{grDevices}: contains all the code implementing the various
graphics devices, including X11, PDF, PostScript, PNG, etc.
\end{itemize}
\end{frame}

\begin{frame}{The Process of Making a Plot}
  When making a plot one must first make a few choices (not
  necessarily in this order):
\begin{itemize}
\item To what device will the plot be sent?  The default in Unix is
  \code{x11}; on Windows it is \code{windows}; on Mac OS X it is
  \code{quartz}
\item Is the plot for viewing temporarily on the screen, or will it
  eventually end up in a paper?  Are you using it in a presentation?
  Plots included in a paper/presentation need to use a file device
  rather than a screen device.
\item
Is there a large amount of data going into the plot?  Or is it just a
few points?  
\item
Do you need to be able to resize the graphic?  
\end{itemize}
\end{frame}


\begin{frame}{The Process of Making a Plot}
\begin{itemize}
\item
What graphics system will you use: base or grid/lattice?  These
generally cannot be mixed.
\item
Base graphics are usually constructed piecemeal, with each aspect of
the plot handled separately through a series of function calls; this
is often conceptually simpler and allows plotting to mirror the
thought process
\item
Lattice/grid graphics are usually created in a single function call,
so all of the graphics parameters have to specified at once;
specifying everything at once allows R to automatically calculate the
necessary spacings and font sizes.
\end{itemize}
\end{frame}


\begin{frame}{Base Graphics}
Base graphics are used most commonly and are a very powerful system
for creating 2-D graphics.  
\begin{itemize}
\item
Calling \code{plot(x, y)} or \code{hist(x)} will launch a graphics
device (if one is not already open) and draw the plot on the device
\item
If the arguments to \code{plot} are not of some special class, then
the \textit{default method} for \code{plot} is called; this function
has \textit{many} arguments, letting you set the title, x axis lable,
y axis label, etc.
\item
The base graphics system has \textit{many} parameters that can set and
tweaked; these parameters are documented in \code{?par}; it wouldn't
hurt to memorize this help page!
\end{itemize}
\end{frame}

\begin{frame}{Some Important Base Graphics Parameters}
The \code{par} function is used to specify global graphics parameters
that affect all plots in an R session.  These parameters can often be
overridden as arguments to specific plotting functions.
\begin{itemize}
\item
pch:  the plotting symbol (default is open circle)
\item
lty: the line type (default is solid line), can be dashed, dotted,
etc.
\item
lwd: the line width, specified as an integer multiple
\item
col: the plotting color, specified as a number, string, or hex code;
the \code{colors} function gives you a vector of colors by name
\item
las: the orientation of the axis labels on the plot
\end{itemize}
\end{frame}

\begin{frame}{Some Important Base Graphics Parameters}
\begin{itemize}
\item
bg: the background color
\item
mar:  the margin size
\item
oma:  the outer margin size (default is 0 for all sides)
\item
mfrow: number of plots per row, column (plots are filled row-wise)
\item
mfcol: number of plots per row, column (plots are filled column-wise)
\end{itemize}
\end{frame}

\begin{frame}[fragile]{Some Important Base Graphics Parameters}
Some default values.
\begin{verbatim}
> par("lty")
[1] "solid"
> par("lwd")
[1] 1
> par("col")
[1] "black"
> par("pch")
[1] 1
\end{verbatim}
\end{frame}

\begin{frame}[fragile]{Some Important Base Graphics Parameters}
Some default values.
\begin{verbatim}
> par("bg")
[1] "transparent"
> par("mar")
[1] 5.1 4.1 4.1 2.1
> par("oma")
[1] 0 0 0 0
> par("mfrow")
[1] 1 1
> par("mfcol")
[1] 1 1
\end{verbatim}
\end{frame}

\begin{frame}{Some Important Base Plotting Functions}
\begin{itemize}
\item
\code{plot}: make a scatterplot, or other type of plot depending on
the class of the object being plotted
\item
\code{lines}: add lines to a plot, given a vector x values and a
corresponding vector of y values (or a 2-column matrix); this function
just connects the dots
\item
\code{points}: add points to a plot
\item
\code{text}: add text labels to a plot using specified x, y
coordinates
\item
\code{title}: add annotations to x, y axis labels, title, subtitle,
outer margin
\item
\code{mtext}: add arbitrary text to the margins (inner or outer) of
the plot
\item
\code{axis}: adding axis ticks/labels
\end{itemize}
\end{frame}


\begin{frame}{Useful Graphics Devices}
The list of devices is found in \code{?Devices}; there are also
devices created by users on CRAN
\begin{itemize}
\item
\code{pdf}: useful for line-type graphics, vector
format, resizes well, usually portable
\item
\code{postscript}: older format, also vector format and resizes well,
usually portable, can be used to create encapsulated postscript files,
Windows systems often don't have a postscript viewer
\item
\code{xfig}: good of you use Unix and want to edit a plot by hand
\end{itemize}
\end{frame}

\begin{frame}{Useful Graphics Devices}
\begin{itemize}
\item
\code{png}: bitmapped format, good for line drawings or images with
solid colors, uses lossless compression (like the old GIF format),
most web browsers can read this format natively, good for plotting
many many many points, does not resize well
\item
\code{jpeg}: good for photographs or natural scenes, uses lossy
compression, good for plotting many many many points, does not resize
well, can be read by almost any computer and any web browser, not
great for line drawings
\item
\code{bitmap}: needed to create bitmap files (png, jpeg) in certain
situations (uses Ghostscript), also can be used to create a variety of
other bitmapped formats not mentioned
\item
\code{bmp}: a native Windows bitmapped format
\end{itemize}
\end{frame}


\begin{frame}{Copying Plots}
There are two basic approaches to plotting.
\begin{enumerate}
\item
Launch a graphics device
\item
Make a plot; annotate if needed
\item
Close graphics device
\end{enumerate}
Or
\begin{enumerate}
\item
Make a plot on a screen device (default); annotate if needed
\item
Copy the plot to another device if necessary (not an exact process)
\end{enumerate}
\end{frame}


\begin{frame}{Copying Plots}
Copying a plot to another device can be useful because some plots
require a lot of code and it can be a pain to type all that in again
for a different device.
\begin{itemize}
\item \code{dev.copy}: copy a plot from one device to another
\item \code{dev.copy2pdf}: copy a plot to a Portable Document Format
  (PDF) file
\item \code{dev.list}: show the list of open graphics devices
\item \code{dev.next}: switch control to the next graphics device on
  the device list
\item \code{dev.set}: set control to a specific graphics device
\item \code{dev.off}: close the current graphics device
\end{itemize}
NOTE: Copying a plot is not an exact operation!
\end{frame}


\begin{frame}{Lattice Functions}
\begin{itemize}
\item
\code{xyplot}: this is the main function for creating scatterplots
\item
\code{bwplot}: box-and-whiskers plots (``boxplots'')
\item
\code{histogram}: histograms
\item
\code{stripplot}: like a boxplot but with actual points
\item
\code{dotplot}: plot dots on ``violin strings''
\item
\code{splom}: scatterplot matrix; like \code{pairs} in base graphics
system
\item
\code{levelplot}, \code{contourplot}: for plotting ``image'' data
\end{itemize}
\end{frame}


\begin{frame}[fragile]{Lattice Functions}
Lattice functions generally take a formula for their first argument,
usually of the form
\begin{verbatim}
y ~ x | f * g
\end{verbatim}
\begin{itemize}
\item
On the left of the \verb+~+ is the y variable, on the right is the x
variable
\item
After the \verb+|+ are \textit{conditioning variables} --- they are
optional; the \verb+*+ indicates an interaction
\item
The second argument is the data frame or list from which the variables
in the formula should be obtained.
\item
If no data frame or list is passed, then the parent frame is used.
\item
If no other arguments are passed, there are defaults that can be used.
\end{itemize}
\end{frame}


\begin{frame}{Lattice Behavior}
Lattice functions behave differently from base graphics functions in
one critical way.
\begin{itemize}
\item
Base graphics functions plot data directly the graphics device
\item
Lattice graphics functions return an object of class \code{trellis}.
\item
The print methods for lattice functions actually do the work of
plotting the data on the graphics device.
\item
Lattice functions return ``plot objects'' that can, in principle, be
stored (but it's usually better to just save the code + data).
\item
On the command line, \code{trellis} objects are \textit{auto-printed}
so that it appears the function is plotting the data
\end{itemize}
\end{frame}


%% \begin{frame}[fragile]{Calling Lattice Functions}
%% \begin{verbatim}
%% p <- xyplot(y ~ x | f, subscripts = TRUE,
%%             ylim = lattice:::extend.limits(range(lo, hi)),
%%             panel = function(x, y, subscripts, ...) {
%%                 panel.xyplot(x, y, ...)
%%                 lsegments(x, lo[subscripts], 
%%                           x, hi[subscripts])
%%                 panel.abline(h = 0, lty = 2)
%%             },
%%             xlab = NULL, 
%%             scales = list(x = list(at = 1:nmodels, labels = models, 
%%                                    rot = 90, alternating = FALSE), 
%%                           y = list(alternating = 3)),
%%             ylab = list(label = expression("% increase in
%%             admissions for a 10 " * mu * g/m^3 * " inceas%% e in " * PM[2.5]), cex = 0.8)
%%             )
%% print(p)
%% \end{verbatim}
%% \end{frame}


\begin{frame}[fragile]{Lattice Panel Functions}
Lattice functions have a \code{panel} function which controls what
happens inside each panel of the entire plot.
\begin{verbatim}
x <- rnorm(100)
y <- x + rnorm(100, sd = 0.5)
f <- gl(2, 50, labels = c("Group 1", "Group 2"))
xyplot(y ~ x | f)
\end{verbatim}
plots y vs. x conditioned on f.
\end{frame}

\begin{frame}[fragile]{Lattice Panel Functions}
\begin{verbatim}
xyplot(y ~ x | f,
       panel = function(x, y, ...) {
               panel.xyplot(x, y, ...)
               panel.abline(h = median(y),
                            lty = 2)
       })
\end{verbatim}
plots y vs. x conditioned on f with horizontal (dashed) line drawn at
the median of y for each panel.
\end{frame}


\begin{frame}[fragile]{Lattice Panel Functions}
Adding a regression line
\begin{verbatim}
xyplot(y ~ x | f,
       panel = function(x, y, ...) {
               panel.xyplot(x, y, ...)
               panel.lmline(x, y, col = 2)
       })
\end{verbatim}
fits and plots a simple linear regression line to each panel of the
plot.
\end{frame}

\begin{frame}[fragile]{Using Subscripts}
Sometimes you need to access objects outside the panel envrionment.
\begin{verbatim}
y <- c(rnorm(10), rnorm(10, 2))
x <- rep(1:10, 2)
std <- rep(1, 20)
rng <- range(y - 1.96 * std, y + 1.96 * std)
f <- gl(2, 10)

xyplot(y ~ x | f, subscripts = TRUE, ylim = rng,
       panel = function(x, y, subscripts, ...) {
               panel.xyplot(x, y, ...)
               lsegments(x, y - 1.96 * std[subscripts],
                         x, y + 1.96 * std[subscripts])
               panel.abline(h = 0, lty = 2)
       })
\end{verbatim}
\end{frame}


\begin{frame}{Mathematical Annotation}
R can produce \LaTeX-like symbols on a plot for mathematical
annotation.  This is very handy and is useful for making fun of people
who use other statistical packages.
\begin{itemize}
\item
Math symbols are ``expressions'' in R and need to be wrapped in the
\code{expression} function
\item
There is a set list of allowed symbols and this is documented in ?plotmath
\item
Plotting functions that take arguments for text generally allow
expressions for math symbols
\end{itemize}
\end{frame}


\begin{frame}[fragile]{Mathematical Annotation}
Some examples.
\begin{verbatim}
plot(0, 0, main = expression(theta == 0), 
     ylab = expression(hat(gamma) == 0),
     xlab = expression(sum(x[i] * y[i], i==1, n)))
\end{verbatim}
Pasting strings together.
\begin{verbatim}
x <- rnorm(100)
hist(x, 
     xlab=expression("The mean (" * bar(x) * ") is " * 
                     sum(x[i]/n,i==1,n)))
\end{verbatim}
\end{frame}

\begin{frame}[fragile]{Substituting}
What if you want to use a computed value in the annotation?
\begin{verbatim}
x <- rnorm(100)
y <- x + rnorm(100, sd = 0.5)
plot(x, y, 
     xlab=substitute(bar(x) == k, list(k=mean(x))),
     ylab=substitute(bar(y) == k, list(k=mean(y)))
     )
\end{verbatim}
Or in a loop of plots
\begin{verbatim}
par(mfrow = c(2, 2))
for(i in 1:4) {
        x <- rnorm(100)
        hist(x, main=substitute(theta==num,list(num=i)))
}
\end{verbatim}
\end{frame}


\begin{frame}{Summary of Important Help Pages}
\begin{itemize}
\item
?par
\item
?plot
\item
?xyplot
\item
?plotmath
\item
?axis
\end{itemize}
\end{frame}


\end{document}
